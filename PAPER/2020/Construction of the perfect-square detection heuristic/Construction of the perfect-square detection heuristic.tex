\documentclass{article}
\usepackage[utf8]{inputenc}
\usepackage{kotex}
\usepackage{fancyhdr}

\pagestyle{fancy}
\fancyhf{}
\lhead{Construction of the perfect-square detection heuristic}
\rhead{ANEP Research}
\rfoot{Page \thepage}

\begin{document}
    \begin{titlepage}
        \centering
	    %\includegraphics[width=0.15\textwidth]{example-image-1x1}\par\vspace{1cm}
	    {\scshape\LARGE Kangwon Science Highschool \par}
	    \vspace{1cm}
	    {\scshape\Large A novel $O(|P|k\max_{p_i \in P}{p_i})$-time heuristic\par}
        \vspace{1.5cm}
	    {\huge\bfseries Construction of the perfect-square number detection heuristic\par}
	    \vspace{2cm}
	    {\Large\itshape Kim Jun Hyeok\par}
	    \vfilla
	    {\large \today\par}
    \end{titlepage}
    \linespread{1.5}
    \begin{abstract}
        기존에 이분 탐색과 산술(Arithmetic) 연산을 이용하여 제시되었던 $k$-bit 정수에 대하여 $O(k^{3})$혹은 $O(k^{2}\log{k})$시간 결정론적 알고리즘은 매우 큰 수에 대해서는 시간이 오래 걸리는 결점이 있다.
        이 연구에서는, 빅데이터 분석을 통해서 임의의 홀수인 소수에 관한 이차 잉여(Quadratic-residue)에 대하여 Euler’s criterion 을 이용한 완전 제곱수 휴리스틱(Heuristic) 제시와 유의미한 높은 확률로 판정 가능한 소수들의 최소 개수를 제시하는 것을 목표로 한다. 
        또한, 후속 연구 방안으로 임의의 제곱수를 비트로 표현하였을 때 임의의 개수의 제곱수에 관하여 LCS(Longest-common- sequence)를 구하여 여기에서 찾은 LCS 를 통해서 매칭하는 휴리스틱을 제시하고, 위의 방법과 결합하여 더욱 판정률을 높인다.
    \end{abstract}
    \tableofcontents
    \newpage
    \section[도입]{Introduction}
        기존에 제시된 결정론적 다항 시간(Deterministic polynomial time)에 임의의 $k$-bit 정수가 제곱수(Perfect-square number)인지 결정하는 알고리즘은, 이분 탐색(Binary search)와 산술 연산(Arithmetic operation)을 통해서 $O(k^{2})$ Naïve 정수 곱셈을 이용하면, $O(k^{3})$시간에 결정할 수 있다.
        여기에서, Fürer 의 $O(k\log{k})$ 시간 정수 곱셈 알고리즘을 이용하면, $O(k^{2}\log{k})$시간에 문제를 해결할 수 있다. [1]
        또한, 임의의 홀수인 소수 $p$에 대하여 다음과 같은 Euler’s criterion 을 통해서 $a \in GF(p)$에 대하여 Legendre’s symbol 을 계산할 수 있다. [2]
        여러 소수에 대하여 반복적으로 시도해봄으로서 높은 확률로 제곱수임을 판정할 수 있다.
        본 연구에서는 임의의 $k$-bit 정수가 제곱수(Perfect-square number)인지 결정하는 $O(k\log{k})$시간 휴리스틱을 제시한다.
\end{document}