\documentclass{article}
\usepackage[utf8]{inputenc}
\usepackage{kotex}
\usepackage{fancyhdr}
\usepackage{blindtext}
\usepackage{amsmath}
\usepackage{amssymb}
\usepackage{hyperref}

\pagestyle{fancy}
\fancyhf{}
\lhead{Quantum Computing 개념지도 보고서}
\rhead{Kim Jun Hyeok}
\rfoot{Page \thepage}

\begin{document}
    \begin{titlepage}
        \centering
	    %\includegraphics[width=0.15\textwidth]{example-image-1x1}\par\vspace{1cm}
	    {\scshape\LARGE Kangwon Science Highschool \par}
        \vspace{1.5cm}
	    {\huge\bfseries Quantum Computing 개념지도 보고서\par}
	    \vspace{2cm}
	    {\Large\itshape Kim Jun Hyeok\par}
	    \vfill
	    {\large \today\par}
    \end{titlepage}
    \linespread{1.5}
    \newpage
    \section{개념}
    \subsection{양자 컴퓨팅의 기초}
        \paragraph{Overview}
        \begin{itemize}
            \item \href{https://en.wikipedia.org/wiki/Oracle_machine}{Oracle} 
            \item \href{https://en.wikipedia.org/wiki/Black_box}{Black-box} 
            \item \href{https://en.wikipedia.org/wiki/Qubit}{Qubit}
            \item \href{https://en.wikipedia.org/wiki/Bloch_sphere}{Bloch Sphere}
            \item \href{https://en.wikipedia.org/wiki/Quantum_logic_gate}{Quantum Gates}
        \end{itemize}
        \paragraph{Quantum Gates}
        \begin{itemize}
            \item Hadamard Gate
            \item SWAP Gate
            \item Pauli-X Gate
            \item Pauli-Y Gate
            \item Pauli-Z Gate
            \item Controlled-X Gate
            \item Phase Shift Gate
        \end{itemize}
    \subsection{양자 알고리즘}
        \paragraph{Quantum Fourier Transform}
        고전적인 DFT는 다음과 같다.
        \begin{equation}
            y_{k} = \frac{1}{\sqrt{N}}{\sum_{n=0}^{N-1} x_{n}\omega_{N}^{-kn}}
        \end{equation}
        이를 큐비트에 대하여 맞추어 변형하면, 결국 각각의 비트 기저에 대하여 Tensor-product 형태로 변형이 되니
        이러한 변환을 $O(\log{n}^{2})$ 시간에 수행할 수 있다. \href{https://en.wikipedia.org/wiki/Quantum_Fourier_transform}{참고 자료}
        \paragraph{Grover's algorithm}
        $O(f(n))$시간에 수행되는 모든 탐색 문제를 높은 확률로 $O(\sqrt{f(n)})$ 시간에
        해결하는 알고리즘이다. \href{https://en.wikipedia.org/wiki/Grover%27s_algorithm}{참고 자료}
        \paragraph{Shor's algorithm}
        소인수 분해 문제를 $\mathbb{Z}/N\mathbb{Z}$의 주기성을 이용하여 주기를 QFT로 구하여 $O(\log{n}^{2})$ 시간에 해결하는 알고리즘. \href{https://en.wikipedia.org/wiki/Shor%27s_algorithm}{참고 자료}
        \paragraph{Deutsch–Jozsa algorithm}
        임의의 $f: {0,1}^{n} \to {0,1}$에 대하여 이 함수를 큐비트의 각각의 비트에 대하여 계산할때, $O(1)$시간에 수행할 수 있는 알고리즘. \href{https://en.wikipedia.org/wiki/Deutsch%E2%80%93Jozsa_algorithm}{참고 자료}
    \section{최근 논문}
    \subsection{TSP 문제를 양자 컴퓨팅으로 해결}
    TSP 문제를 다항시간에 양자 컴퓨팅으로 해결하는 논문.
    \href{https://arxiv.org/abs/1805.10928}{arXiv:1805.10928}
    \subsection{Vertex cover 문제를 양자 컴퓨팅으로 해결}
    Vertex cover 문제를 다항시간에 양자 컴퓨팅으로 해결하는 논문.
    \href{https://arxiv.org/abs/2009.06726}{arXiv:2009.06726}
\end{document}
